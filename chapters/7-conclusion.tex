\chapter{Conclusion}
\label{chap:conclusion}

\iffalse
Minioppsummering, med fokus på konsekvenser, konkludere om du har besvart RQ.

Her skal det ikke introduseres noe nytt! Trekker sammen de viktige tingene til slutt.

Inneholder også future work, her kan du utdype litt, men ingenting "nytt" her.


The primary contribution of this project is an improved method for correlating Windows Event Logs in time, in near real time. The goal of this thesis is to explore ways to improve real time log correlation both performance-wise but also addressing the problems that occur when analyzing asynchronous events or when experiencing log ingestion delays.
\fi
The goals of this thesis was to outlined the state of the art in real time event correlation, and implemented a solution that improves the way real time event correlation can be done with regards to Windows Event log correlation. This solution builds upon existing solutions.. \n{elaborate}

First of all we did a deep dive into the state of the art and considered several relevant types of event correlation. Rule-based event correlation was chosen for SEVERAL REASONS \n{add some reasons}, and relevant opponents was identified. \n{elaborate}

A implementation was created that utilized the same rule-set as SEC. We implemented multi-threading which saw a great effect. \n{elaborate}

A new implementation was proposed that uses different rules for correlating events. Two different methods for context management was explored as part of this. \n{elaborate}

The experiments and the associated results present the event processing and correlation throughout which showed a varying level of increased performance, depending on the dataset and methods used for context management. The highest speed improvement seen was up to 135\%. \n{elaborate}

In conclusion, this thesis has outlined the state of the art in real time event correlation, and implemented a solution that improves the way real time event correlation can be done with regards to Windows Event log correlation.
Different implementations have been created and tested for performance through experiments using datasets that are both realistic, and optimized for testing performance. The experiments served as proof-of-concept that we were able to enhance and improve the event processing throughput and correctness compared to existing solutions. As a result, this thesis has made a contribution to event correlation, and more specifically for correlating Windows Event logs in near real time.