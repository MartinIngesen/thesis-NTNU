
% From https://www.overleaf.com/learn/latex/Glossaries

\makeglossaries % Prepare for adding glossary entries


\newglossaryentry{latex}
{
        name=latex,
        description={Is a mark up language specially suited for
scientific documents}
}

\newglossaryentry{bibliography}
{
        name=bibliography,
        plural=bibliographies,
        description={A list of the books referred to in a scholarly work,
typically printed as an appendix}
}

\newglossaryentry{maths}
{
    name=mathematics,
    description={Mathematics is what mathematicians do}
}


% --------------------
% ----- Acronyms -----
% --------------------

\newacronym{api}{API}{Application Programming Interface}
\newacronym{av}{AV}{Anti-Virus}
\newacronym{hids}{HIDS}{Host-based Intrusion Detection System}
\newacronym{nsm}{NSM}{Network Security Monitoring}
\newacronym{ids}{IDS}{Intrusion Detection System}
\newacronym{ips}{IPS}{Intrusion Prevention System}
\newacronym{sec}{SEC}{Simple Event Correlator}
\newacronym{sysmon}{Sysmon}{System Monitor}
\newacronym{siem}{SIEM}{Security Information and Event Management}
\newacronym{xml}{XML}{Extensible Markup Language}
\newacronym{gpo}{GPO}{Group Policy Object}
\newacronym{ddos}{DDoS}{Distributed Denial-of-Service}
\newacronym{fsm}{FSM}{Finite-state machine}
\newacronym{sql}{SQL}{Structured Query Language}
\newacronym{yaml}{YAML}{YAML Ain't Markup Language}

